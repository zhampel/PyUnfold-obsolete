

%%%%%%%%%%%%%%%%%%% Simple Unfolding Section %%%%%%%%%%%%%%%%%%

\subsection{PyUnfold: Unfolding Made Simple - But Not Simpler}
The PyUnfold package is available for checkout on \href{https://github.com/zhampel/PyUnfold}{GitHub},
and was used in the spectrum analysis found in \cite{hawc-cr-spectrum}.
Its main dependencies are \verb|matplotlib|, \verb|pyROOT|, and \verb|numpy|.
It is rather simple to implement a spline regularization routine, simply ask the author how.
To run an unfolding from the command line is simple:

\mint{bash}|python Unfold.py -c config.cfg -p|

This will unfold a data set with response matrix, prior, and other input parameters all specified in the configuration file (via the \verb|-c| flag).
An output ROOT \cite{root} file, also specified in \verb|config.cfg|, will be generated with all pertinent details of the unfolding, 
including the input effects distribution, all cause distribution iterations, and the test statistic and regularization parameter values for each iteration.
The \verb|-p| flag plots the input effects distribution, a subset of unfolded distribution iterations, and the final unfolded cause distribution.


One can call the unfolding function from within other python scripts as well.
This can be useful if one desires to unfold an effects distribution subject to different response functions,
or perhaps different input distributions with the same response matrix, for example. 
To implement this, the \verb|Unfold.py| script is first imported, then the main unfolding function can be called as:

%\begin{minted}[frame=lines,framerule=1pt]{python}
\begin{minted}[frame=single,framerule=1pt]{python}
# import the unfolding project and utils
from Unfold import Unfold
from Utils import *

# grab the effects distribution 
# with errors and axis properties 
# as numpy arrays here

# set the RecoDist object here to pass to unfolder
RecoDist = DataDist("Reco",data=data,error=error,axis=cause,edges=cause_edges,
                     xlabel="Reco Var",ylabel="Freq.",units="Arb Units")

# do the unfolding
CauseDist = Unfold(config_name="config.cfg",return_dists=True,EffDist=RecoDist)

# grab the unfolded distribution and errors
unf_dist = CauseDist.getData()
unf_err = CauseDist.getError()

\end{minted}
Of course, in this example, the script would be located in the PyUnfold project directory in order to import \verb|Unfold| and \verb|Utils|.
For clarity, \verb|RecoDist| is an instance of the class \verb|DataDist| found in \verb|Utils.py|, which gets passed to the \verb|Unfold| function. 
The input variables \verb|data|, \verb|error|, \verb|axis|, \verb|edges| are all assigned to numpy arrays containing their respective namesakes. 
Hence, the bin centers for the \verb|data| distribution are defined by \verb|axis|, and the bin edges by \verb|edges|. 
Finally, the unfolded distribution and respective error numpy arrays are accessed as shown in the final two lines of the above example.

If one specifically is making an {\bf energy} dependent flux measurement, the \verb|FluxFit.py| script can be used to account for 
the energy bin spacing to calculate a differential flux from the unfolded counts distribution. The command command line with all options is

\mint{bash}|python FluxFit.py -c config.cfg -p -s 2.6 -w|

where \verb|-p| plots the differential flux scaled by the value given to \verb|-s|, and the resulting flux values and fit parameters (function defined in the config file)
are written to the same unfolding output ROOT file with \verb|-w|.


%%%%%%%%%%%%%%%%%%% ROOT File Section %%%%%%%%%%%%%%%%%%
\subsection{ROOT}
A quick word on ROOT. The PyUnfold package relies on some tools provided by ROOT and pyROOT.
These are {\bf requirements} to using PyUnfold.

Another requirement is the structure of ROOT files.
As shown in the next section, data structures are stored as histograms in a specific directory structure.
When generating a ROOT file with the efficiency histogram and response matrix, they {\bf must} be located inside 
a directory labeled \verb|binX| where \verb|X| is an integer identifying the bin number.
Similarly, the corresponding data set to unfold must be in a directory with the same name in its ROOT file.
This ensures one can use different analysis cuts defining differnet bins and pack them orderly in respective data and response files.
Apart from this, one can name the ROOT objects as one wishes, so long as these labels are reflected in the configuration file.

For two analysis bins, the directories for the three ROOT files defined in the configuration file follow this structure:

% Latex file structure code taken from:
% http://tex.stackexchange.com/questions/5073/making-a-simple-directory-tree

\begin{multicols}{3}
\begin{forest}
  for tree={
    font=\sffamily, minimum height=0.75cm, rounded corners=4pt, grow'=0, inner ysep=8pt, child anchor=west,parent anchor=south,
    anchor=west, calign=first, edge={rounded corners},
    edge path={
      \noexpand\path [draw, \forestoption{edge}]
      (!u.south west) +(12.5pt,0) |- (.child anchor)\forestoption{edge label};
    },
    before typesetting nodes={
      if n=1
        {insert before={[,phantom,minimum height=18pt]}}
        {}
    },
    fit=band, s sep=2pt, before computing xy={l=25pt},
  }
  [stats\_file
    [\myfolder{bin0}
      [eff\_hist]
      [mm\_hist]
    ]
    [\myfolder{bin1}
      [eff\_hist]
      [mm\_hist]
    ]
  ]
\end{forest}

\begin{forest}
  for tree={
    font=\sffamily, minimum height=0.75cm, rounded corners=4pt, grow'=0, inner ysep=8pt, child anchor=west,parent anchor=south,
    anchor=west, calign=first, edge={rounded corners},
    edge path={
      \noexpand\path [draw, \forestoption{edge}]
      (!u.south west) +(12.5pt,0) |- (.child anchor)\forestoption{edge label};
    },
    before typesetting nodes={
      if n=1
        {insert before={[,phantom,minimum height=18pt]}}
        {}
    },
    fit=band, s sep=2pt, before computing xy={l=25pt},
  }
  [input\_file
    [\myfolder{bin0}
      [ne\_meas]
      [obstime]
    ]
    [\myfolder{bin1}
      [ne\_meas]
      [obstime]
    ]
  ]
\end{forest}

\begin{forest}
  for tree={
    font=\sffamily, minimum height=0.75cm, rounded corners=4pt, grow'=0, inner ysep=8pt, child anchor=west,parent anchor=south,
    anchor=west, calign=first, edge={rounded corners},
    edge path={
      \noexpand\path [draw, \forestoption{edge}]
      (!u.south west) +(12.5pt,0) |- (.child anchor)\forestoption{edge label};
    },
    before typesetting nodes={
      if n=1
        {insert before={[,phantom,minimum height=18pt]}}
        {}
    },
    fit=band, s sep=12pt, before computing xy={l=25pt},
  }
  [outfile
    [\myfolder{{\footnotesize \ UNF\_TS\_COV}}
      [\myfolder{bin0}
        [nc\_meas]
        ]
      [\myfolder{bin1}
        [nc\_meas]
        ]
      ]
    ]
  ]
\end{forest}


\end{multicols}

%%%%%%%%%%%%%%%%%%% Configuration File Section %%%%%%%%%%%%%%%%%%

\subsection{Configuration File}

The configuration file contains many keyword sections and options, however, only a few parameters need specification to get started.
The following sections are from a complete config file which is included in the GitHub project. 
Each header section is presented in detail below.
For beginners, the most relevant sections are \verb|data|, \verb|mcinput|, \verb|output|, and \verb|regularization|.
Other sections have default values set for most purposes.

\newpage
\noindent\hrulefill

\inputminted[firstline=5,lastline=11]{ini}{config_example.cfg}
The \verb|data| section corresponds to the extracted data effects distribution.
In practice, one will only modify the first two options:
the path to the ROOT input data file (\verb|inputfile|), and the name of the measured effects histogram to unfold (\verb|ne_meas|).
The \verb|obstime| option identifies the ROOT histogram containing a single bin the total observation time in seconds.
If one is unfolding a Monte Carlo generated data set, then the \verb|ismc| flag must be set to \verb|True| and the appropriate cause distribution (\verb|nc_thrown|)
can be identified.

\noindent\hrulefill

\inputminted[firstline=13,lastline=17]{ini}{config_example.cfg}
The \verb|mcinput| section corresponds to the response function model. 
The \verb|stats_file| key identifies the path to the ROOT file containing the histograms for the 1D efficiency, \verb|eff_hist|, and 2D mixing matrix, \verb|mm_hist|.


\noindent\hrulefill

\inputminted[firstline=19,lastline=22]{ini}{config_example.cfg}
The \verb|output| section details whether to write the output to file (\verb|write_output|), which file to write (\verb|outfile|), and the name
of the final unfolded cause distribution (\verb|nc_meas|).

\noindent\hrulefill

\inputminted[firstline=1,lastline=3]{ini}{config_example.cfg}
The \verb|analysisbins| section tells the unfolder which bin(s) to consider for the unfolding. 
The \verb|bin| option specifies the analysis bin number to access in the above ROOT files.
The value must be an integer, and the unfolder expects the ROOT files to have the directory \verb|bini| for the $i$th bin.
For example, one could define a set of bins each with different data cuts, and thus require respectively different response functions.
The \verb|data| \verb|inputfile| would also hold the extracted data distributions in their respective bins.
Of course, we can only unfold a single {\bf data} bin at a time. Thus, for most unfoldings, \verb|bin| will be a single integer.

If one desires to stack several bins, for example to include response functions from two distinct cause populations,
then the appropriate bin numbers are provided to \verb|bin| via comma separated values. The \verb|stack| flag also must be set to \verb|True|.
In this case, the \verb|ne_meas| in the \verb|data| \verb|inputfile| {\bf must} be located in the \verb|bin0| directory, though this is not explicitly
stated in the configuration file.

A quick note on stacked unfolding: there is nothing new in introducing a stacked unfolding. Looking back to Section \ref{unfolding_method_section},
we see that the definition of the causes, $C_{\mu}$, is completely general. Hence, one can simply `stack' families of causes together.
Visually, this is equivalent to setting two response matrices next to each other along the cause axis.
For example, we can split the all-particle cosmic ray spectrum into two groups: light (P+He) and heavy (> He). 
Their respective response functions are generated separately (with the same analysis cuts) and placed in different bin directories of the same \verb|stats_file|.
A {\bf single} effects distribution is extracted from the data, again with the same cuts, and must be in \verb|bin0| of the \verb|data| \verb|inputfile|.
When running the unfolder, the results for each stacked bin are saved in respective directories of the same \verb|outfile|.

In general one must be mindful of the choice of initial priors for each bin when stacking. The simplest example is to stack the same response matrix with itself.
If we choose priors for bins $0,1$ with fixed ratio $r$ for all cause bins, ie

\begin{equation} \label{eq:equalstackedpriors}
 \begin{split}
    \frac{P_{0}(C_{\mu})}{P_{1}(C_{\mu})} = r \quad \forall \mu.
 \end{split}
\end{equation}
then the unfolded cause distributions will retain this ratio through all iterations. 
Of course, summing the two distributions at any point in the unfolding gives the same result as using the single response matrix.


\noindent\hrulefill

\inputminted[firstline=24,lastline=27]{ini}{config_example.cfg}
Here the user can define a functional form of the initial prior.
The default is `Jeffreys' prior \cite{jeffreys}.
A user provided function must be single-valued and conform to \verb|numpy| syntax if special functions are used.
If doing a stacked unfolding, formulae must be separated by commas for each bin.


\noindent\hrulefill

\inputminted[firstline=29,lastline=34]{ini}{config_example.cfg}
This section gives the user choice over the name of the unfolding function (\verb|unfolder_name|), 
the maximum allowed number of iterations via \verb|max_iter|,
and whether to include regularization at each step with \verb|smooth_with_reg|.
Details of the regularization are found in the next section.
The \verb|verbose| flag prints out detailed information of the unfolding.



\noindent\hrulefill

\inputminted[firstline=36,lastline=48]{ini}{config_example.cfg}
The choice of regularization scheme can be manipulated in a manner consistent with ROOT histogram fitting.
The functional form in \verb|reg_func| is single-valued and the parameters must be labeled in sequential order via [].
The restrictions on range in x is defined with \verb|reg_range|. 
For example, when unfolding the all-particle spectrum, regularization to a power law is done on a subrange of interest in Energy.
The fit parameters are given labels, initial values, and low and high limits per the respective keywords.
Finally, for detailed fitting information, \verb|plot| and \verb|verbose| flags can be set appropriately.


\noindent\hrulefill

\inputminted[firstline=50,lastline=57]{ini}{config_example.cfg}
The test statistic, $TS$, is used to compare the unfolded distributions from two consecutive iterations: $TS = TS(N_{i},N_{i+1})$.
This threshold to determine convergence is defined with the \verb|ts_tolerance| keyword,
and a subset of the cause range can be provided via \verb|ts_range|.
The test statistic functions available are the following:

\begin{itemize}
  \item rmd: Maximum relative deviation, $\sup(\frac{|N_{i}-N_{i+1}|}{N_{i}+N_{i+1}})$
  \item chi2: Reduced $\chi^{2}$
  \item pf: Bayes Factor (Pfendner Factor) \cite{pfenderfactor}
  \item ks: Kolmogorov-Smirnov
\end{itemize}


\noindent\hrulefill

\inputminted[firstline=59,lastline=62]{ini}{config_example.cfg}
The \verb|mixer| section simply names the mixer function and has two methods for the calculation 
of the covariance matrix due to the limited statistics in the response function.

\begin{itemize}
  \item DCM: D'Agositini method outlined in \cite{agostini}
  \item ACM: Adye uncertainty from \cite{adye2}; defined in this document with Eqs. \ref{eq:Vn_matrix} and \ref{eq:Vmc_matrix} and fully outlined in Section \ref{adye_full_derivation}
\end{itemize}

While the comment in this section claims that DCM is faster, the updated implementation of ACM no longer influences performance.


\noindent\hrulefill

\inputminted[firstline=64,lastline=80]{ini}{config_example.cfg}
The final section is identical in form to the \verb|regularization| section.
However, it only pertains to fitting a differential flux once the unfolding procedure is finished.
The parameters here are only read by the \verb|FluxFit.py| script, and hence
can be modified to test different fit parameters and/or functions after the unfolding.
